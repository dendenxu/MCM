%%%%%%%%%%%%%%%%%%%%%%%%%%%%%%%%%%%%%%%%
%% MCM/ICM LaTeX Template %%
%% 2020 MCM/ICM           %%
%%%%%%%%%%%%%%%%%%%%%%%%%%%%%%%%%%%%%%%%
\documentclass[12pt]{article}
% \documentclass{article}
\usepackage{geometry}
\geometry{left=1in,right=0.75in,top=1in,bottom=1in}

%%%%%%%%%%%%%%%%%%%%%%%%%%%%%%%%%%%%%%%%
% Replace ABCDEF in the next line with your chosen problem
% and replace 1111111 with your Team Control Number
\newcommand{\Problem}{\textcolor{cyan}{REPLACE ME WITH PROBELM}}
\newcommand{\Team}{\textcolor{cyan}{REPLACE ME WITH TEAM NUMBER}}
%%%%%%%%%%%%%%%%%%%%%%%%%%%%%%%%%%%%%%%%

% \usepackage{newtxtext}
\usepackage{amsmath,amssymb,amsthm}
\usepackage{newtxmath} % must come after amsXXX

\usepackage[pdftex]{graphicx}
\usepackage{xcolor}
\usepackage{fancyhdr}
\usepackage{setspace}
\usepackage{amsmath}
\usepackage{graphicx}
\usepackage{float}
\usepackage{subcaption}
% This is gonna give you footnote
% \usepackage[backend=bibtex,style=verbose-trad2]{biblatex}
% And this is gonna give you some collection
\usepackage[backend=bibtex,style=ieee]{biblatex}
\usepackage{siunitx} % Required for alignment
\usepackage{multirow}
\usepackage{booktabs} % For prettier tables
\usepackage{longtable} % To display tables on several pages
\usepackage{rotating} % To display tables in landscape
\usepackage{pgfplotstable} % Generates table from .csv
\usepackage{tikz}
\usepackage{pgfplots}
\usepackage{csvsimple}
\usepackage{listings}
\usepackage[utf8]{inputenc}
% Well guess you need this to make sure line break of minted listing is working fine
\usepackage[newfloat]{minted}
% \usepackage{caption}
\usepackage{pgf}
\usepackage{import}
\usepackage{pdfpages}
\bibliography{summary}


\lhead{Team \Team}
\rhead{}
\cfoot{}

\newtheorem{theorem}{Theorem}
\newtheorem{corollary}[theorem]{Corollary}
\newtheorem{lemma}[theorem]{Lemma}
\newtheorem{definition}{Definition}

%%%%%%%%%%%%%%%%%%%%%%%%%%%%%%%%
 \begin{document}
\graphicspath{{.}}  % Place your graphic files in the same directory as your main document
\DeclareGraphicsExtensions{.pdf, .jpg, .tif, .png}
\thispagestyle{empty}
\vspace*{-16ex}
\centerline{\begin{tabular}{*3{c}}
        \parbox[t]{0.3\linewidth}{\begin{center}\textbf{Problem Chosen}\\ \Large \textcolor{red}{\Problem}\end{center}}
         & \parbox[t]{0.3\linewidth}{\begin{center}\textbf{2020\\ MCM/ICM\\ Summary Sheet}\end{center}}
         & \parbox[t]{0.3\linewidth}{\begin{center}\textbf{Team Control Number}\\ \Large \textcolor{red}{\Team}\end{center}} \\
        \hline
    \end{tabular}}
%%%%%%%%%%% Begin Summary %%%%%%%%%%%
% Enter your summary here replacing the (red) text
% Replace the text from here ...
\begin{center}
    \textcolor{red}{%
        Use this template to begin typing the first page (summary page) of your electronic report. This \newline
        template uses a 12-point Times New Roman font. Submit your paper as an Adobe PDF \newline
        electronic file (e.g. 1111111.pdf), typed in English, with a readable font of at least 12-point type.	\\[2ex]
        Do not include the name of your school, advisor, or team members on this or any page.	\\[2ex]
        Papers must be within the page limit specified in the problem statement.	\\[2ex]
        Be sure to change the control number and problem choice above.	\\
        You may delete these instructions as you begin to type your report here. 	\\[2ex]
        \textbf{Follow us @COMAPMath on Twitter or COMAPCHINAOFFICIAL on Weibo for the \newline
            most up to date contest information.}
    }
\end{center}
% to here
%%%%%%%%%%% End Summary %%%%%%%%%%%

%%%%%%%%%%%%%%%%%%%%%%%%%%%%%%
\clearpage
\pagestyle{fancy}
% Uncomment the next line to generate a Table of Contents
\tableofcontents
\newpage
\setcounter{page}{1}
\rhead{Page \thepage\ }

\section{Introduction}
\subsection{Problem Background}
\subsection{Our Work}

\section{Assumptions \& Nomenclature}
\subsection{Assumptions}
\subsection{Nomenclature}


\section{Modeling Under Waves and Tides}
\par
The 3-dimensional shape constructing problem is divided into two subproblems. We first establish the model using the \textbf{Mohr-Coulomb Criterion} to decide the shape of the slope. We then construct another model with a modified version of \textbf{Cellular Automata} to determine the best shape viewed from the top,...

\subsection{Shape of the Slope: Mohr-Coulomb Criterion}
\par
We begin by determining the Side shape of the sand castle fundation. Before Approaching the problem, we will briefly address the property of sand as a granular media.
\par
In our assumptions, sand particle are considered as a simplified model of identical tiny spheres. If we zoom in to observe a pile of wet sand, there are the so-called liquid bridges formed between sand particles.
\par
Various water contents produce different liquid bridge distributions, which will influence the properties of sand. 

\begin{table}[H]
	\caption{States of Sand}
	\vspace{10pt}
	\centering
	\begin{tabular}{p{2cm}p{3cm}p{2.5cm}p{2.5cm}p{2.5cm}p{2.5cm}}
		\hline
		Liquid Content & State & Description & left for future use \\
		\hline
		No  			  & Dry   	    & 000          & 000                    \\
		Small  			  & Pendular    & 000     	   & 000                        \\
		Middle  		  & Funicular   & 000    	   & 000                        \\
		Almost saturated  & capillary & 000          & 000                        \\
		More  			  & Slurry      & 000          & 000                        \\  
		\hline       
	\end{tabular}
	\label{bs2}
\end{table}
\par
When the wave hits the foundation, the surface area is in slurry state and there exists no cohesive interaction between the particles, which makes it very hard, if not impossible, to prevent sand loss. Nevertheless, collapses after the wave resides can cause more harm to the foundation, which can be avoided by alternating the shape.
\par 
For dry sand, the failure criterion is given in terms of the shear stress $\tau$, the normal compressible stress $\sigma$ and the internal friction $\mu$ as
$$\tau > \mu\sigma$$
This is simply the friction formula with different notations. Now we consider a sandpile with a normal adhesive stress $s_A$ across every plane, in addition to the stress caused by weight. The equation(1) is then modified as
$$\tau > \mu(\sigma + s_A)$$
This criterion is the so-called \textbf{Mohr-Coulumb criterion}.The stress resulting from the weight above the plane is shown in figure (1). Denote $\tau_f$ and $\mu_f$ as the shear stress and normal compressible stress at the failure plane,they can be written as
$$\tau_f = \rho gDsin\theta_c \qquad and \qquad \sigma_f = \rho gDcos\theta_c$$
where $\theta_c$ is the critical angle, $D$ is the height of the sandpile and $\rho$ is the density of sand. Therefore, to solve for $\theta_c$ is to solve the equation
$$\mu = tan\theta_c(D)\bigg(1 + \frac{s_A}{\rho gDcos\theta_c(D)}\bigg)$$
\par
The only unknow factor is $s_A$, the adhesive stress across the plane. According to (Thomas C.H and Alex J.L -fix later)'s study, the value of $s_A$ is determined by water content and there are three regimes as a function of the added-fluid volume. We now focus only on the state where the water content is close to saturated. 

\subsection{Top View Shape}

\subsection{Calculating Results}
\subsection{Simulating Results}
\subsection{Refer to Footnote} 

\section{Modeling Under Rain}

\section{Determine the Best Sand-to-water Proportion}
\section{Other Ways to Make Our Sand Castle Last Longer}

\section{Sensitivity Analysis}
 
This is a random citation \autocite{LeeRice-4} here.
% This is a random citation \cite{LeeRice-4} here.
And this would be another citation: \autocite{AragonRios-30}.
% And this would be another citation: \cite{AragonRios-30}.
Here's another \autocite{Starobin-32} one.
% Here's another \cite{Starobin-32} one.

\subsection{Footnote}
% If you use BibLaTex you may also use the default footnote of latex
Random citation \autocite{LeeRice-4} embedded in text.
This is some example text\footnote{\label{myfootnote}Hello footnote}.

\subsection{Refer to Footnote}
I'm referring to footnote \ref{myfootnote}.

\newpage
\begin{appendix}
    \listoffigures
    \listoftables
    \listoflistings
    % \bibliographystyle{ieeetr}
    \printbibliography
\end{appendix}
\end{document}
